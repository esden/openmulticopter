%          File: /documentation/mini/v0.1/hardware/hardware.tex
%       Created: Tue Mar 02 03:00 PM 2010 C
%   Last Change: Tue Mar 02 03:00 PM 2010 C
%
%   Copyright (c) by . All rights reserved.
%

% Use for english
\documentclass[11pt, a4paper]{article}

\usepackage{a4wide}
\usepackage[T1]{fontenc}
\usepackage{times}
\usepackage[utf8]{inputenc}

% Header / Footer
\usepackage{lastpage}
\usepackage{fancyhdr}
\pagestyle{fancy}
% Clean all default header/footer
\fancyhf{}
\headheight 25pt

% Header
\fancyhead[L]{openmulticopter - mini}
\fancyhead[C]{hardware}
\fancyhead[R]{\leftmark}

% Footer
\fancyfoot[L]{\today}
\fancyfoot[C]{openmulticopter.org}
\fancyfoot[R]{Page \thepage\ / \pageref{LastPage}}


% Document information
\title{openmulticopter mini - hardware v0.1}
\author{}
\date{\today}

% Surround parts of graphics with box
\usepackage{boxedminipage}

% Package for including code in the document
\usepackage{listings}

% Additional symbols
\usepackage{textcomp}

% Color package
\usepackage{color}
\usepackage[table]{xcolor}

% Math package
\usepackage{amsmath}

% Graphics support
\usepackage{graphics}

% No indentation
\setlength{\parindent}{0in}

% This is now the recommended way for checking for PDFLaTeX:
\usepackage{ifpdf}

\ifpdf
\usepackage[pdftex]{graphicx}
\else
\usepackage{graphicx}
\fi
%===================Start document====================

\begin{document}

% Display title etc
\maketitle

\begin{abstract}
This document describes the different hardware parts/units from the openmulticopter-mini version.
It covers mostly calculations and technical aspects of this project.
\end{abstract}

\newpage

\tableofcontents
\newpage

\section{Summary} % (fold)
\label{sec:Summary}
The mini version of the openmulticopter is a first version of our project. It’s only for boot strapping 
our work and for testing/development purposes. As simple as possible. And of course, as small as possible.
% section Summary (end)

\section{Design Ideas} % (fold)
\label{sec:Design Ideas}
\begin{itemize}
  \item Use the internal ADC of the STM32 for sensor inputs
  \item JTAG programming interface exposed
  \item I2C/CAN interface exposed for motor controller interfacing
  \item PPM input for sum signal receivers
  \item UART interface for DSL receivers (RCOPEN24 support?)
  \item UART interface exposed for PC/RF modem interface
  \item Some LEDs for status indication
  \item AUX connector containing as many of the remaining pins of the STM32 as possible and some of 
    the already exposed pins (for further tests and expansions) 
  \item \dots
\end{itemize}

% section Design Ideas (end)

\section{Selected main-parts} % (fold)
\label{sec:Selected Main-Parts}

\begin{center}\footnotesize
  \definecolor{light-gray}{gray}{0.95}
  \definecolor{normal-gray}{gray}{0.85}
  \rowcolors{2}{light-gray}{normal-gray}

  \begin{tabular}{ | l | p{6.5cm} | l | l |}
    \hline Part & Description & Name & Footprint \\ \hline
    Controller & 
    High-density Performance line, ARM-based 32-bit MCU with up to 512 KB Flash, USB, CAN, 
    11 timers, 3 ADCs, 13 communication interfaces & 
    ST STM32F104RE & 
    TQFP64 \\ \hline

    Gyros x, y, z (melexis) & 
    Factory set full scale range: ±75 °/s, ±150 °/s or ±300 °/s. 5V supply
    voltage. Analog / digital interface. Output not supply-ratiometric. &
    MLX90609 &
    CQFN32 \\ \hline

    Gyros x, y, z (analog devices) &
    High vibration rejection over wide frequency. Ratiometric to referenced supply. 5V single-supply
    operation &
    ADXRS610 &
    BG-32-3 \\ \hline

    Accelerometer &
    3-Axis +/- 1.5g accelerometer, single-supply operation: 2.7-3.6V, ratiometric ouptuts &
    MXR9500 &
    LCC16 \\ \hline

    I2C level-shifter &
    Hot swappable dual I2C isolators, bidirectional I2C communication, 1000 kHz operation, 3.0V to
    5.5V supply/logic levels &
    ADUM1250 &
    SOIC-8 \\ \hline

    CAN transceiver &
    High-speed CAN transceiver, supports 1Mb/s operation, suitable for 12V and 24V systems,
    detection of ground fault (permanent dominant) on TXD input &
    MCP2551 &
    SOIC-8 \\ \hline

    Switchting regulator &
    Recom 78xx series. 5V output &
    &
    78xx \\ \hline
  \end{tabular}
\end{center}


% section Selected Main-Parts (end)

% = = = = = = Circuitry
\section{Circuitry description} % (fold)
\label{sec:Circuitry description}

\subsection{Power Supply} % (fold)
\label{sub:Power Supply}

% sub Power Supply (end)

\subsection{I2C level shifter} % (fold)
\label{sub:I2C level shifter}

% sub I2C level shifter (end)

\subsection{CAN interface} % (fold)
\label{sub:CAN interface}

% sub CAN interface (end)

\subsection{Accelerometer} % (fold)
\label{sub:Accelerometer}

% sub Accelerometer (end)

\subsection{Gyroscopes} % (fold)
\label{sub:Gyroscopes}

% subsection Gyroscopes (end)

\subsection{STM32} % (fold)
\label{sub:STM32}

% sub STM32 (end)
% section Circuitry description (end)

% = = = = = = Connectors
\section{Connectors} % (fold)
\label{sec:Connectors}

\subsection{Extension board} % (fold)
\label{sub:Extension board}

% sub Extension board (end)

\subsection{RC-receiver} % (fold)
\label{sub:RC-receiver}

% sub RC-receiver (end)

\subsection{USART interface} % (fold)
\label{sub:USART interface}

% sub USART interface (end)

\subsection{Motor controllers} % (fold)
\label{sub:Motor controllers}

% sub Motor controllers (end)

\subsection{JTAG interface} % (fold)
\label{sub:JTAG interface}

% sub JTAG interface (end)
% section Connectors (end)

\end{document}

